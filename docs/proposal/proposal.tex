\documentclass{article}

\usepackage{sectsty}
\usepackage{amssymb}
\usepackage{amsmath}
\usepackage{graphicx}

\begin{document}

% title stuff
\title{Interactive Fractal Viewer \\
CSEE W4840 Project Proposal}
\author{Luis E. P. - lep2141@columbia.edu \and
Nathan Hwang - nyh2105@columbia.edu \and
Stephen Pratt - sdp2128@columbia.edu \and
Richard Nwaobasi - rcn2105@columbia.edu}
\maketitle

% ------------------------------------------------------------
\section{Description/Goals}

We wish to create a device that will generate and display quadratic
polynomial Julia Set fractals. Members of a quadratic polynomial Julia
Set are points on the complex plane which stay bounded in a sequence
of iterated function evaluations where the function $f$ is of the form:

\begin{equation}
f(z) = z^2 + c
\end{equation}

Where $z$ is the complex number in question and $c$ is some function parameter.

Typically, images of Julia sets are colorized according to how long it
takes for each point on the plane to become unbounded (pass some
distance threshold). We would like to perform this same colorization.

Note that there a few parameters that define the device’s
behavior. These include:

\begin{itemize}
\item The constant $c$ that uniquely describes the Julia Set
  generated. (complex number)
\item The horizontal range of the display window. (range in the reals)
\item The vertical range of the display window. (range in the reals)
\item The colorization function $k$. (function mapping positive integers
  to an RGB color value)
\end{itemize}

We are certain that we would like the user to be able select the
constant $c$ from a number of pre-selected constants using the builtin
switches on the Altera DE2 board. Window settings might be handled
similarly, by way of the push buttons on the board, using a peripheral
such as a keyboard, or not at all. Our ambition is to cycle through
colorization functions by way of audio-in processing, but we
recognize that this feature will be both time consuming and auxiliary.

% ------------------------------------------------------------
\section{Mission-Critical Modules}

\begin{itemize}
\item Iterated function module (IFM) takes complex number as input,
  applies function iteratively, and then produces an output describing
  (a) if the function stayed bounded or (b) for how many iterations it
  stayed bounded if it became unbounded
\item Generator module with two jobs:
  \begin{itemize}
  \item Keep track of which pixels still need to be
    generated. Associate those pixels with a complex number determined
    by our data window and display resolution
  \item Ensure that each pixel gets sent off to an IFM for
    processing. Two caveats here:
    \begin{itemize}
    \item Multiple IFMs can and should work in parallel
    \item The processing time for each pixel varies
    \end{itemize}
  \end{itemize}
\item Intermediate display buffer (ID Buffer) maps each pixel passed
  along by the Generator module to the output of the IFM for that
  pixel.
\item Colorization module takes each value from the ID Buffer, applies
  the colorization function $k$ to the value, and stores the resulting
  RGB value $k(n)$ at a point in a frame buffer corresponding to the
  point in the ID buffer from whence the value came.
\item Frame buffer stores RGB values for each pixel.
\item VGA out module reads from the frame buffer and produces a VGA
  output signal.
\end{itemize}

% ------------------------------------------------------------
\section{Parametrization Modules}
Time permitting their implementation, these modules can modulate the
output in interesting ways.

\begin{itemize}
\item The Seed Module maps switch configurations to a Julia Set
  constant c. The module must communicate this value to each IFM, and
  ask the generator module for a remapping.
\item Display selection module determines the x range and y range of
  the display window.  The module must communicate these ranges to the
  generator module and ask for a remapping.
\item K-selector module cycles through colorization functions and
  reports the function to the colorization module.
\item Spectral analysis module takes audio input and uses it to effect
  the performance of the k-selector module.
\end{itemize}

% ------------------------------------------------------------
\section{Implementation Ideas of Mission-Critical Modules}

\begin{itemize}
\item IFM: floating point multiplication, addition, and comparison loop
\item Generator Module: State, integer addition, interface with IFM
\item ID Buffer: Encoder to turn Pixels to Addresses, RAM
\item Colorization Module: Fancy decoder
\item Frame Buffer: RAM
\item VGA Out: VGA module a la Lab 3
\end{itemize}

% ------------------------------------------------------------
\section{Milestones}

\begin{itemize}
\item Milestone 0 (ASAP): \\
  Be able to add and multiply floating point numbers.
\item Milestone 1 (Mar 27): \\
  Have a static Julia set filled into the ID buffer.
\item Milestone 2 (Apr 10): \\
  Display the colorized Julia set through VGA.
\item Milestone 3 (Apr 24): \\
  Implement parameter changing, with
  subsequent updates to the displayed Julia set.
\end{itemize}

\end{document}
